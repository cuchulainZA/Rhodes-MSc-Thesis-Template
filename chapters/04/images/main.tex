%%%%%%%%%%%%%%%%%%%%%%%%%%%%%%%%%%%%%%%%%%%%%%%%%
%%%%%%%%%%%%%%%%%%%%%%%%%%%%%%%%%%%%%%%%%%%%%%%%%

\chapter{Experimental Setup}

\label{chap:exp}

\section{Random maths}
An example equation: 

\begin{equation}
    \label{eq:OpinionTriangle}
    \begin{aligned}
        b+d+u=1, \{b,d,u\}\in[0,1]
    \end{aligned}
\end{equation}

\begin{equation}
    \label{eq:OpinionGeneral}
    \begin{array}{l l}
    w_p^{A} & =\{b_p^A,d_p^A,u_p^A\} \\
    \text{where,}
    \end{array}
\end{equation}
\begin{equation}
    \label{eq:OpinionCalc}
     w_A^B = \left\{
    \begin{array}{l l}
        b_A^B \dfrac{p}{p+n+u} & \quad \\
        d_A^B \dfrac{n}{p+n+u} & \quad \text{, where } u_A^B \neq  0\\
        u_A^B \dfrac{u}{p+n+u} &
    \end{array} \right.
\end{equation}

An example algorithm: 
\begin{algorithm}
    \label{alg:Lorenz}
	\centering{\fbox{\parbox[]{155mm}{
		\begin{algorithmic}[1]
    	\Require W[ ]- a sorted array of all wealth distributions
    	\Function{LorenzCurve}{W[ ]}
    	
    	   \State $L[ ] $ \Comment{An empty set of equal length to W[ ]}
    	    
    	   \State $currentTotal \gets 0$
            \For {$i\gets 1,W.length()$ }
            
                \State $currentTotal \gets currentTotal + W[i]$
                
                \State $L[i] \gets currentTotal$
            \EndFor
            
            \Return $L[  ]$
        \EndFunction
    \end{algorithmic}
	}}}
	\caption{Pseudo code for calculating Lorenz Curve plot points}
	%\label{alg:Lorenz}
\end{algorithm}


%%%%%%%%%%%%%%%%%%%%%%%%%%%%%%%%%%%%%%%%%%%%%%%%%
%%%%%%%%%%%%%%%%%%%%%%%%%%%%%%%%%%%%%%%%%%%%%%%%%
And here you can see an image of a rarely spotted special snowflake in Figure \ref{fig:snowflake}
\begin{figure}[H]
    \centering
    \includegraphics[width=88mm]{chapters/03/images/snow_flake}
    \caption{Special snowflake}
    \label{fig:snowflake}
\end{figure}